\section{Weak Pre Shared Key, Key Authentication (WPSKKA) Protocol}

The WPSKKA protocol is the E2EE Encryption layer protocol used to communicate between Peers. All WPSKKA messages'
first byte contain a number to indicate
the message type.\\

WPSKKA relies on SRP as defined in \href{https://datatracker.ietf.org/doc/html/rfc5054}{RFC5054} to establish a
shared key used to authenticate DH keys via a mac. The Host will serve as the SRP server, the Client will serve as the SRP
client.\\

\begin{center}
    Host \textrightarrow\ Client\\
    \begin{tabular}{|c|c|c|}
        \hline
        \textbf{Bytes} & \textbf{Name} & \textbf{Value} \\
        \hline
        1              & type          & 1              \\
        \hline
        16             & username      &                \\
        \hline
        16             & salt          &                \\
        \hline
        256            & srp-B         &                \\
        \hline
        16             & public-key    &                \\
        \hline
    \end{tabular}
\end{center}

\begin{align*}
    & (D_{host}^{pub},\, D_{host}^{priv}) := \text{DH-Generate()}\\
    & I := RAND(128)\\
    & s := RAND(128)\\
    & B := \text{SRP-B()}\\
    & \text{username} := I\\
    & \text{salt} := s\\
    & \text{srp-B} := B\\
    & \text{public-key} := D_{host}^{pub}\\
\end{align*}

\begin{center}
    Client \textrightarrow\ Host\\
    \begin{tabular}{|c|c|c|}
        \hline
        \textbf{Bytes} & \textbf{Name} & \textbf{Value} \\
        \hline
        1              & type          & 2              \\
        \hline
        256            & srp-A         &                \\
        \hline
        16             & public-key    &                \\
        \hline
        32             & mac           &                \\
        \hline
    \end{tabular}
\end{center}

\begin{align*}
    & (D_{client}^{pub},\, D_{client}^{priv}) := \text{DH-Generate()}\\
    & \text{srp-A} := \text{SRP-A()}\\
    & L_{client} = L_{host} := \text{SRP-PREMASTER()}\\
    & \text{public-key} := D_{client}^{pub}
    & \text{mac} := \text{HMAC}(\text{KDF}_1(D_{client}^{pub}, L_{client}))\\
\end{align*}

\begin{center}
    Host \textrightarrow\ Client\\
    \begin{tabular}{|c|c|c|}
        \hline
        \textbf{Bytes} & \textbf{Name} & \textbf{Value} \\
        \hline
        1              & type          & 3              \\
        \hline
        32             & mac           &                \\
        \hline
    \end{tabular}
\end{center}

\begin{align*}
    & \text{mac} := \text{HMAC}(\text{KDF}_1(D_{host}^{pub}, L_{host}))\\
\end{align*}

\subsection{Transport Data Key Derivation}

\begin{align*}
    & Q_{host} = \text{DH}(D_{client}^{pub},\ D_{host}^{priv})\\
    & Q_{client} = \text{DH}(D_{host}^{pub},\ D_{client}^{priv})\\
    & (U_{peer}^{send} = U_{serv}^{recv},\ U_{peer}^{recv} = U_{serv}^{send}) := \text{KDF}_2(Q_{host} = Q_{client},
    \ \epsilon) \\
    & O_{host}^{send} = O_{client}^{recv} = O_{host}^{recv} = O_{client}^{send} := 0
\end{align*}

\subsubsection{Subsequent Messages: Transport Data Messages}

\begin{center}
    Host $\leftrightarrow$ Client\\
    \begin{tabular}{|c|c|c|}
        \hline
        \textbf{Bytes}     & \textbf{Name} & \textbf{Value} \\
        \hline
        1                  & type          & 4              \\
        \hline
        8                  & counter       &                \\
        \hline
        2                  & data-length   &                \\
        \hline
        \emph{data-length} & data          &                \\
        \hline
    \end{tabular}
\end{center}

\begin{align*}
    & \text{data} := \text{AEAD}(U_{m}^{send}, O_{m}^{send}, P, \epsilon)\\
    & \text{counter} := O_{m}^{send}\\
    & O_{m}^{send} := O_{m}^{send} + 1
\end{align*}


Where \emph{P} is the payload to be transported.

$O_{m}$ is an 64 bit counter that MUST NOT wrap. After a transport message is sent, if $O_{m}$ equals
($2^{64}-1$) the TCP connection MUST be dropped. Subsequent messages MUST NOT be sent. \\

