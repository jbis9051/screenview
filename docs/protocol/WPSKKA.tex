\section{Weak Pre Shared Key, Key Authentication (WPSKKA) Protocol}

The WPSKKA protocol is the E2EE Encryption layer protocol used to communicate between Peers. All WPSKKA messages'
first byte contain a number to indicate
the message type.\\

\subsection{Handshake}

The handshake begins with a key exchange.
Then then the Client is offered a list of available authentication schemes.
The goal of WPSKKA is to authenticate the Peers' ephemeral keys through one of those schemes.

\subsubsection{KeyExchange}

The Host and Client exchange their ephemeral public keys.

\begin{center}
    Host $\leftrightarrow$ Client\\
    \begin{tabular}{|c|c|c|}
        \hline
        \textbf{Bytes} & \textbf{Name} & \textbf{Value} \\
        \hline
        1              & type          & 1              \\
        \hline
        32             & public-key    &                \\
        \hline
    \end{tabular}
\end{center}

\subsubsection{AuthScheme}

Once the Host receives the Client's public key, it sends the Client a list of authentication schemes it supports.

\begin{center}
    Host \textrightarrow\ Client\\
    \begin{tabular}{|c|c|c|}
        \hline
        \textbf{Bytes}   & \textbf{Name}    & \textbf{Value} \\
        \hline
        1                & type             & 2              \\
        \hline
        1                & num-auth-schemes &                \\
        \hline
        num-auth-schemes & auth-schemes     &                \\
        \hline
    \end{tabular}
\end{center}

num-auth-schemes is the number of auth schemes.
auth-schemes contains 1 byte per auth-scheme to indicate which authentication scheme is available.
Authentication schemes are defined below.\\

\begin{center}
    Authentication Schemes\\
    \begin{tabular}{|c|c|}
        \hline
        \textbf{Number} & \textbf{Name} \\
        \hline
        0               & None          \\
        \hline
        1               & SRP Dynamic   \\
        \hline
        2               & SRP Static    \\
        \hline
        3               & Public key    \\
        \hline
    \end{tabular}
\end{center}

\subsubsection{TryAuth}

The TryAuth messages indicates a Client would to attempt authentication with a particular auth scheme.

\begin{center}
    Client \textrightarrow\ Host\\
    \begin{tabular}{|c|c|c|}
        \hline
        \textbf{Bytes} & \textbf{Name}      & \textbf{Value} \\
        \hline
        1              & type               & 3              \\
        \hline
        1              & auth-scheme-number &                \\
        \hline
    \end{tabular}
\end{center}

auth-scheme-number is the authentication scheme the Client would like to attempt.
If the Host does not support the auth-scheme-number sent by the Client, the Host MUST send a failed AuthResult message.

\subsubsection{AuthMessage}

Messages used in the authentication scheme are encapsulated in this message.

\begin{center}
    Host $\leftrightarrow$ Client\\
    \begin{tabular}{|c|c|c|}
        \hline
        \textbf{Bytes} & \textbf{Name} & \textbf{Value} \\
        \hline
        2              & length        &                \\
        \hline
        1              & type          & 4              \\
        \hline
        length-1       & auth-message  &                \\
        \hline
    \end{tabular}
\end{center}

\subsubsection{AuthResult}

The AuthResult message indicates the result of the authentication attempt.
A successful authentication result MUST only be sent once.

\begin{center}
    Host $\rightarrow$ Client\\
    \begin{tabular}{|c|c|c|}
        \hline
        \textbf{Bytes} & \textbf{Name} & \textbf{Value} \\
        \hline
        2              & length        & 2              \\
        \hline
        1              & type          & 5              \\
        \hline
        1              & ok            & 0 or 1         \\
        \hline
    \end{tabular}
\end{center}

ok indicates whether the authentication method was successful (1) or failed (0).
If the authentication method failed, the Client MAY attempt another authentication scheme or try the same one again by sending a TryAuth message.

\subsection{Transport Data Key Derivation}

\begin{align*}
    & C_{host} = \text{DH}(E_{client}^{pub},\ E_{host}^{priv})\\
    & C_{client} = \text{DH}(E_{host}^{pub},\ E_{client}^{priv})\\
    & (\mathit{ST}_{host}^{send} = \mathit{ST}_{client}^{recv},\ \mathit{ST}_{host}^{recv} = \mathit{ST}_{client}^{send}, \mathit{SU}_{host}^{send} = \mathit{SU}_{client}^{recv},\ \mathit{SU}_{host}^{recv} = \mathit{SU}_{client}^{send}) \coloneqq \text{KDF}_4(C_{host} = C_{client},
    \ \epsilon) \\
    & \mathit{NT}_{host}^{send} = \mathit{NT}_{client}^{recv} = \mathit{NT}_{host}^{recv} = \mathit{NT}_{client}^{send} = \mathit{NU}_{host}^{send} = \mathit{NU}_{client}^{recv} = \mathit{NU}_{host}^{recv} = \mathit{NU}_{client}^{send} \coloneqq 0
\end{align*}

ST keys and NT nonces are used for TCP. SU keys and TU nonces are using for UDP.

\subsection{Subsequent Messages: Transport Data Messages}

Transport Data Messages MUST NOT be sent until authentication is complete.

\subsubsection{TCP}

\begin{center}
    Host $\leftrightarrow$ Client\\
    \begin{tabular}{|c|c|c|}
        \hline
        \textbf{Bytes}  & \textbf{Name} & \textbf{Value} \\
        \hline
        2               & data-length   &                \\
        \hline
        1               & type          & 6              \\
        \hline
        remaining bytes & data          &                \\
        \hline
    \end{tabular}
\end{center}

\begin{align*}
    & \text{data} \coloneqq \text{AEAD}(\mathit{ST}_{m}^{send},\mathit{NT}_{m}^{send}, P, \epsilon)\\
    & \text{counter} \coloneqq \mathit{NT}_{m}^{send}\\
    & \mathit{NT}_{m}^{send} \coloneqq \mathit{NT}_{m}^{send} + 1
\end{align*}


Where \emph{P} is the payload to be transported.\\

$\mathit{NT}_{m}$ is an 64 bit counter that MUST NOT wrap. After a transport message is sent, if $\mathit{NT}_{m}$ equals
($2^{64}-1$) the UDP and TCP connection MUST be dropped. Subsequent messages MUST NOT be sent. \\

\subsubsection{UDP}

\begin{center}
    Host $\leftrightarrow$ Client\\
    \begin{tabular}{|c|c|c|}
        \hline
        \textbf{Bytes}  & \textbf{Name} & \textbf{Value} \\
        \hline
        2               & data-length   &                \\
        \hline
        1               & type          & 7              \\
        \hline
        8               & counter       &                \\
        \hline
        remaining bytes & data          &                \\
        \hline
    \end{tabular}
\end{center}

\begin{align*}
    & \text{data} \coloneqq \text{AEAD}(\mathit{SU}_{m}^{send},\mathit{NU}_{m}^{send}, P, \epsilon)\\
    & \text{counter} \coloneqq \mathit{NU}_{m}^{send}\\
    & \mathit{NU}_{m}^{send} \coloneqq \mathit{NU}_{m}^{send} + 1
\end{align*}


Where \emph{P} is the payload to be transported.\\

$\mathit{NU}_{m}$ is an 64 bit counter that MUST NOT wrap. After a transport message is sent, if $\mathit{NU}_{m}$ equals
($2^{64}-1$) the UDP and TCP connection MUST be dropped. Subsequent messages MUST NOT be sent. \\

\subsection{SRP Dynamic/SRP Static}

SRP relies on SRP as defined in \href{https://datatracker.ietf.org/doc/html/rfc5054}{RFC5054} to establish a
shared key used to authenticate the ephemeral public keys via a MAC.

The Host will serve as the SRP server, the Client will serve as the SRP client.
The password will be created and stored by the Host.
This password can be randomly generated (Dynamic) or chosen by the Host (Static).
The Host may serve both Dynamic and Static SRP authentication methods.
The Client will select which type to use in the TryAuth message.
\\

\subsubsection{HostHello}

The Host sends a HostHello message to the Client.

\begin{center}
    Host \textrightarrow\ Client\\
    \begin{tabular}{|c|c|c|}
        \hline
        \textbf{Bytes} & \textbf{Name} & \textbf{Value} \\
        \hline
        1              & type          & 1              \\
        \hline
        16             & username      &                \\
        \hline
        16             & salt          &                \\
        \hline
        256            & srp-B         &                \\
        \hline
    \end{tabular}
\end{center}

\begin{align*}
    & I \coloneqq RAND(128)\\
    & s \coloneqq RAND(128)\\
    & B \coloneqq \text{SRP-B()}\\
    & \text{username} \coloneqq I\\
    & \text{salt} \coloneqq s\\
    & \text{srp-B} \coloneqq B\\
\end{align*}

\subsubsection{ClientResponse}

\begin{center}
    Client \textrightarrow\ Host\\
    \begin{tabular}{|c|c|c|}
        \hline
        \textbf{Bytes} & \textbf{Name} & \textbf{Value} \\
        \hline
        1              & type          & 2              \\
        \hline
        256            & srp-A         &                \\
        \hline
        32             & mac           &                \\
        \hline
    \end{tabular}
\end{center}

\begin{align*}
    & \text{srp-A} \coloneqq \text{SRP-A()}\\
    & L_{client} = L_{host} \coloneqq \text{SRP-PREMASTER()}\\
    & \text{mac} \coloneqq \text{HMAC}(\text{KDF}_1(L_{client}), \text{public-key})\\
\end{align*}

public-key is the Client's public key

The Host validates the MAC.
If the MAC is incorrect a failed MUST be AuthResult is sent.

\subsubsection{HostVerify}

\begin{center}
    Host \textrightarrow\ Client\\
    \begin{tabular}{|c|c|c|}
        \hline
        \textbf{Bytes} & \textbf{Name} & \textbf{Value} \\
        \hline
        1              & type          & 3              \\
        \hline
        32             & mac           &                \\
        \hline
    \end{tabular}
\end{center}

\begin{align*}
    & \text{mac} \coloneqq \text{HMAC}(\text{KDF}_1(L_{host}), \text{public-key})\\
\end{align*}

public-key is the Host's public key

After the HostVerify is sent, the Host MUST send a successful AuthResult.

\subsection{Public Key (3)}

WIP