\section{ScreenView Server Communication (SVSC) Protocol }

The SVSC protocol is the Server Communication Layer protocol used for Peers to interact with the relay server,
Server. Peers can lease an ID as well as begin a session with another Peer. Once a session is established, Peers can
forward messages to another Peer. Unless otherwise noted, all messages MUST occur over TCP.\\

With the exception of the \hyperlink{subsection.4.2}{Handshake} messages. All SVSC messages' first byte contain a
number to indicate
the message type.

\subsection{Definitions}

\begin{itemize}
    \item Peer - denotes a client in classical server/client environment
    \item Server - The intermediary server used for routing and proxying data between two Peers
\end{itemize}

\subsection{Handshake}

\subsubsection{ProtocolVersion}

Handshaking begins with the Server sending the Peer a ProtocolVersion message. This lets the server know
the version supported by the Host. the ProtocolVersion message consists of 12 bytes interpreted as a string of ASCII
characters in the format
"SVSC xxx.yyy" where xxx and yyy are the major and minor version numbers, padded with zeros.

\begin{center}
    Server \textrightarrow\ Peer\\
    \begin{tabular}{|c|c|c|}
        \hline
        \textbf{Bytes} & \textbf{Name} & \textbf{Value}            \\
        \hline
        11             & version       & ``\texttt{SVSC 001.000}'' \\
        \hline
    \end{tabular}
\end{center}

The Peer replies back either \texttt{0} to indicate the version is not acceptable and that the handshake has
failed or \texttt{1} if the version is acceptable to the Peer and the handshake as succeeded. If \texttt{0} is sent, all
communication MUST cease and the TCP connection MUST be terminated.

\begin{center}
    Peer \textrightarrow\ Server\\
    \begin{tabular}{|c|c|c|}
        \hline
        \textbf{Bytes} & \textbf{Name} & \textbf{Value} \\
        \hline
        1              & ok            & 0 or 1         \\
        \hline
    \end{tabular}
\end{center}

\subsection{Leasing}

A lease is a temporary assignment of an ID to a Peer. The ID format and generation is discussed in
\hyperlink{subsubsection.4.3.5}{4.3.5}. A maximum of 1 ID can be leased per TCP connection. ID generation MUST be
rate limited to prevent ID exhaustion. Rate limiting rules are out of scope for this protocol, however some
suggestions are listed in \hyperlink{subsubsection.4.3.6}{4.3.6}.

\subsubsection{LeaseRequest}

A \emph{LeaseRequest} message requests a lease of an ID.

\begin{center}
    Peer \textrightarrow\ Server\\
    \begin{tabular}{|c|c|c|}
        \hline
        \textbf{Bytes} & \textbf{Name} & \textbf{Value} \\
        \hline
        1              & type          & 1              \\
        \hline
        1              & has-cookie    & 0 or 1         \\
        \hline
        \multicolumn{3}{|c|}{\textbf{Below only if \emph{has-cookie} is 1} } \\
        \hline
        24             & cookie        &                \\
        \hline
    \end{tabular}
\end{center}

If a Peer would like to request an ID it had previously been issued after expiration, it may include the cookie
value it received in the LeaseResponse. There is no guarantee that the Peer will receive the same ID or that the Server
will even consider the cookie value.

\subsubsection{LeaseResponse}

A LeaseResponse message is a response to a LeaseRequest. If has-cookie is 1, a Server MAY consider the cookie value
in LeaseRequest or completely
ignore it.

\begin{center}
    Server \textrightarrow\ Peer\\
    \begin{tabular}{|c|c|c|}
        \hline
        \textbf{Bytes} & \textbf{Name} & \textbf{Value} \\
        \hline
        1              & type          & 2              \\
        \hline
        1              & accepted      & 0 or 1         \\
        \hline
        \multicolumn{3}{|c|}{\textbf{Below only if \emph{accepted} is 1} } \\
        \hline
        4              & id            &                \\
        \hline
        24             & cookie        &                \\
        \hline
        8              & expiration    &                \\
        \hline
    \end{tabular}
\end{center}

\emph{expiration} is a 64 bit Unix timestamp representing the expiry of lease. Disconnection of a Peer (e.g,
the TCP connection is dropped) does not end a lease.\\

\emph{cookie} a 128 bit value. The generation of this value is discussed in \hyperlink{subsubsection.4.2.7}{4.2
.7}.\\

Consideration of the cookie value MUST have no effect on the the value of accepted. That is, if the
request is for a specific ID (implied by the presence of a cookie value and a has-cookie value equal to 1
in the LeaseRequest) and the ID requested is not available, the Server SHOULD respond with a different
available ID and an accepted value of 1 (assuming an ID is available). accepted MUST only be 0 if no IDs are left,
for rate limiting reasons, or some other reasons unrelated to the cookie value.

\subsubsection{LeaseExtensionRequest}

A LeaseExtensionRequest message is used to extend a lease. Before a lease has expired, the Peer can
request a lease extension. The Server can accept or deny this request. The Peer SHOULD send this message no earlier
than as soon as 50 percent of the lease duration has expired.

\begin{center}
    Peer \textrightarrow\ Server\\
    \begin{tabular}{|c|c|c|}
        \hline
        \textbf{Bytes} & \textbf{Name} & \textbf{Value} \\
        \hline
        1              & type          & 3              \\
        \hline
        24             & cookie        &                \\
        \hline
    \end{tabular}
\end{center}

\subsubsection{LeaseExtensionResponse}

A LeaseExtensionResponse message is a response to a LeaseExtensionRequest.

\begin{center}
    Server \textrightarrow\ Peer\\
    \begin{tabular}{|c|c|c|}
        \hline
        \textbf{Bytes} & \textbf{Name}  & \textbf{Value} \\
        \hline
        1              & type           & 4              \\
        \hline
        1              & extended       & 0 or 1         \\
        \hline
        \multicolumn{3}{|c|}{\textbf{Below only if \emph{extended} is 1} } \\
        \hline
        8              & new-expiration &                \\
        \hline
    \end{tabular}
\end{center}

\emph{new-expiration} is a 64 bit Unix timestamp representing the expiry of lease.

\subsubsection{ID Generation}

An ID is a 26 to 33 bit decimal number. This comes out to about up to 8 to 10 decimal digits, respectively. The
Server may scale the keyspace depending on current usage. For optimal user experience while maintaining
efficiency, the Server MUST only use keyspaces between 26 bits and 33 bits for ID generation. ID generation must
also be uniformly random. All active IDs must be stored on the server. New IDs MUST be unique. ID generation MAY
occur using the below
algorithm:\\

Let $S$ represents a set of all active IDs, $B$ be a number of bits between 26 and 33, and $generate(x)$ be a
functions that returns a $x$ uniformly random bits.

\begin{algorithm}
    \caption{ID generation}
    \begin{algorithmic}
        \State $id$
        \Repeat
            \State $id\gets generate(B)$
        \Until{$id\notin S$}
        \State $S\gets S\cup \{id\}$
        \State \textbf{return} $id$
    \end{algorithmic}
\end{algorithm}

\subsubsection{Rate Limits}

To prevent ID exhaustion, rate limits SHOULD be in place. TCP is used for LeaseRequests so IP addresses
can not be spoofed. However, using proxy services such as Tor, simple IP based rate limits are likely not
entirely sufficient. Servers MAY want to block all known proxy IP addresses.

\subsubsection{Cookie Value}

A \emph{cookie} value is a 128 bit value used for authentication in LeaseExtensionRequest and
LeaseRequest messages. Specific generation of a cookie is out of scope, however care must be taken
to ensure it is not predictable or exploitable. This value MAY be simply a random 24 byte key, HMAC-SHA1($id$,
$key$) $||$ $id$, or something else entirely.

\subsection{Sessions}

A session is a connection between two Peers. At least one Peer must have an ID. A Peer can have a maximum of one
session at any time. Immediately after receiving a EstablishSessionResponse message with a status
of 0 or a EstablishSessionNotification message a Peer MUST establish UDP connection by sending a
Keepalive message as defined in \hyperlink{subsection.4.5}{4.5}. Failure to do so MAY result in dropped
SessionData* packets.

\subsubsection{EstablishSessionRequest}

An EstablishSessionRequest message is a Peer request to establish a session with another Peer.

\begin{center}
    Peer \textrightarrow\ Server\\
    \begin{tabular}{|c|c|c|c|}
        \hline
        \textbf{Bytes} & \textbf{Name} & \textbf{Value} & \textbf{Description}                                 \\
        \hline
        1              & type          & 5              &                                                      \\
        \hline
        4              & lease-id      &                & The ID of the Peer to establish this connection with \\
        \hline
    \end{tabular}
\end{center}

\subsubsection{EstablishSessionResponse}

An EstablishSessionResponse message is a response to EstablishSessionRequest.

\begin{center}
    Server \textrightarrow\ Peer\\
    \begin{tabular}{|c|c|c|c|}
        \hline
        \textbf{Bytes} & \textbf{Name} & \textbf{Value} & \textbf{Description}                       \\
        \hline
        1              & type          & 6              &                                            \\
        \hline
        4              & lease-id      &                & the ID of the Peer attempted to connect to \\
        \hline
        1              & status        & 0-5            & described below                            \\
        \hline
        \multicolumn{4}{|c|}{\textbf{Below only if \emph{status} is 0} } \\
        \hline
        16             & session-id    &                & described below                            \\
        \hline
        16             & peer-id       &                & described below                            \\
        \hline
        16             & peer-key      &                & described below                            \\
        \hline
    \end{tabular}
\end{center}

\emph{status} can have the following values:

\begin{center}
    \begin{tabular}{|c|c|}
        \hline
        \textbf{Value} & \textbf{Description}                 \\
        \hline
        0              & session establishment was successful \\
        \hline
        1              & ID not found                         \\
        \hline
        2              & Peer is offline                      \\
        \hline
        3              & Peer is busy                         \\
        \hline
        4              & You are busy                         \\
        \hline
        5              & Other error                          \\
        \hline
    \end{tabular}
\end{center}

A Peer may be considered offline if, for example, an unexpired ID has been assigned to them and then the TCP
connection is dropped.\\

\emph{session-id} is a 128 bit random value used for session identification\\

\emph{peer-id} is a 128 bit random value used to authentication a Peer for a given session. A Peer's peer-key MUST
never be revealed to anyone but the Peer it belongs to (and the Server that generated it)
for security reasons.\\

\subsubsection{EstablishSessionNotification}

A EstablishSessionNotification notifies a Peer that a session has been established with them.

\begin{center}
    Server \textrightarrow\ Host\\
    \begin{tabular}{|c|c|c|c|}
        \hline
        \textbf{Bytes} & \textbf{Name} & \textbf{Value} & \textbf{Description}                                \\
        \hline
        1              & type          & 7              &                                                     \\
        \hline
        16             & session-id    &                & described in \hyperlink{subsubsection.4.4.2}{4.4.2} \\
        \hline
        16             & peer-id       &                & described in \hyperlink{subsubsection.4.4.2}{4.4.2} \\
        \hline
        16             & peer-key      &                & described in \hyperlink{subsubsection.4.4.2}{4.4.2} \\
        \hline
    \end{tabular}
\end{center}

peer-id and peer-key are the id and key of the Peer being notified NOT the id and key of the Peer they are connecting
to.

\subsubsection{SessionEnd}

A SessionEnd message is used to terminate a session. Once a Server receives a SessionEnd message, the Server
MUST immediately stop forwarding messages and send a SessionEndNotification to the other Peer. The Peer must ignore
any SessionDataPacket message received after this.

\begin{center}
    Peer \textrightarrow\ Server\\
    \begin{tabular}{|c|c|c|c|}
        \hline
        \textbf{Bytes} & \textbf{Name} & \textbf{Value} & \textbf{Description} \\
        \hline
        1              & type          & 8              &                      \\
        \hline
    \end{tabular}
\end{center}

\subsubsection{SessionEndNotification}

A SessionEndNotification notifies a Peer that a session has ended. If a Peer sends a SessionEnd
message, the Server MUST send a SessionEndNotification message to a Peer. The Peer must ignore
any SessionDataPacket message received after this

\begin{center}
    Server \textrightarrow\ Peer\\
    \begin{tabular}{|c|c|c|c|}
        \hline
        \textbf{Bytes} & \textbf{Name} & \textbf{Value} & \textbf{Description} \\
        \hline
        1              & type          & 9              &                      \\
        \hline
    \end{tabular}
\end{center}

\subsubsection{SessionDataSend - TCP/UDP}

A SessionDataSend is a message from a Peer intended to be forwarded to the Peer on the other side of the
session. If a connection is not available (e.g. UDP was dropped or never established) for
SessionDataReceive message to be sent to the other Peer, the SessionDataSend message is silently dropped.

\begin{center}
    Peer \textrightarrow\ Server\\
    \begin{tabular}{|c|c|c|c|}
        \hline
        \textbf{Bytes} & \textbf{Name} & \textbf{Value} & \textbf{Description}           \\
        \hline
        1              & type          & 10             &                                \\
        \hline
        3              & data-length   &                & length of the content in bytes \\
        \hline
        data-length    & data          &                & data to be forwarded           \\
        \hline
    \end{tabular}
\end{center}

\subsubsection{SessionDataReceive - TCP/UDP}

A SessionDataReceive is a message being forwarded to a Peer from the Peer on the other side of the
session. The Server SHOULD forward the message along the same transport as it was received.

\begin{center}
    Server \textrightarrow\ Peer\\
    \begin{tabular}{|c|c|c|c|}
        \hline
        \textbf{Bytes} & \textbf{Name} & \textbf{Value} & \textbf{Description}           \\
        \hline
        1              & type          & 11             &                                \\
        \hline
        3              & data-length   &                & length of the content in bytes \\
        \hline
        data-length    & data          &                & data to be forwarded           \\
        \hline
    \end{tabular}
\end{center}

\subsection{Keepalive - TCP/UDP}

For each transport (TCP and UDP), if no message has been sent in KeepaliveTimeout a Server sends a keepalive
message over the respective transport. The Peer MUST respond with another Keepalive message.\\

For TCP, if a KeepaliveTimeout response is not received by the Server in
double KeepaliveTimeout seconds the TCP connection is considered dropped.\\

For UDP, if a KeepaliveTimeout response is not received by the Server in
half  KeepaliveTimeout seconds another Keepalive message is sent. If a response is not received in
an additional half KeepaliveTimeout seconds, the UDP connection is considered dropped.

\begin{center}
    Server $\leftrightarrow$ Peer\\
    \begin{tabular}{|c|c|c|c|}
        \hline
        \textbf{Bytes} & \textbf{Name} & \textbf{Value} \\
        \hline
        1              & type          & 0              \\
        \hline
    \end{tabular}
\end{center}