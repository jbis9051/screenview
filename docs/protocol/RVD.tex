\section{Remote Visual Display (RVD) Protocol}

The RVD protocol is used to communicate messages regarding mouse input, keyboard input, frame data, and clipboard
data between the
Host and the Client.\\

All messages MUST occur over the transport listed.\\

All RVD messages' first byte contain a number to indicate the message type. \\

\subsection{Definitions}

\begin{itemize}
    \item Host - A peer with an ID that wants to share their screen to the Client
    \item Client - A peer that wants to view and maybe control the Host's screen
    \item Display - A rectangular visual region that is shared by a Host to a Client. May or may not be
    Controllable.
    \item Controllable - A Display that accepts keyboard and mouse input from the Client.
\end{itemize}

\subsection{Handshake}

\subsubsection{ProtocolVersion - TCP}
Handshaking begins by the Host sending the client a ProtocolVersion message. This lets the Client know the
version supported by the Host.\\

The ProtocolVersion message consists of 11 bytes interpreted as a string of ASCII characters in the format
"RVD xxx.yyy" where xxx and yyy are the major and minor version numbers, padded with zeros.

\begin{center}
    Host \textrightarrow\ Client\\
    \begin{tabular}{|c|c|c|}
        \hline
        \textbf{Bytes} & \textbf{Name} & \textbf{Value}           \\
        \hline
        1              & type          & 0                        \\
        \hline
        11             & version       & ``\texttt{RVD 001.000}'' \\
        \hline
    \end{tabular}
\end{center}

The Client replies back either \texttt{0} to indicate the version is not acceptable and that the handshake has
failed or \texttt{1} if the version is acceptable to the Client and the handshake has succeeded.
If 0 is sent, all communication MUST cease and an error SHOULD be displayed to user.
A SessionEnd message should be sent by the Client.

\begin{center}
    Client \textrightarrow\ Host\\
    \begin{tabular}{|c|c|c|}
        \hline
        \textbf{Bytes} & \textbf{Name} & \textbf{Value} \\
        \hline
        1              & type          & 1              \\
        \hline
        1              & ok            & 0 or 1         \\
        \hline
    \end{tabular}
\end{center}

\subsection{Control messages}
Control messages are messages that instruct the Client about changes regarding the Host.

\subsubsection{PermissionsUpdate - TCP}

The PermissionsUpdate message is sent by the Host to the Client to indicate the permissions.
The default permissions are all false.

\begin{center}
    Host \textrightarrow\ Client\\
    \begin{tabular}{|c|c|c|}
        \hline
        \textbf{Bytes} & \textbf{Name} & \textbf{Value} \\
        \hline
        1              & type          & 2              \\
        \hline
        1              & permissions   & see below      \\
        \hline
    \end{tabular}
\end{center}

The bits of permission are defined below in little endian:

\begin{center}
    \begin{tabular}{|c|c|c|}
        \hline
        \textbf{Bit} & \textbf{Name}   \\
        \hline
        0            & clipboard-read  \\
        \hline
        1            & clipboard-write \\
        \hline
        2-7          & reserved        \\
        \hline
    \end{tabular}
\end{center}

\subsubsection{DisplayShare - TCP}
A DisplayShare message informs the Client about a Display the Host would like to share.

For each DisplayShare sent, an additional DisplayShare with the same display-id MUST NOT be sent until a
DisplayUnshare message is sent for that display-id.

The Client MUST respond with a DisplayShareAck message with the same display-id.
The Host MUST NOT send FrameUpdates for the Display until the Host has received a DisplayShareAck message with the display-id.
If the Host does not receive a DisplayShareAck message within 5 seconds, the Host MUST send a DisplayUnshare message
for the display-id and consider the Display unshared.
The Host MAY try to share again.

If the Host receives a DisplayShareAck message with a display-id that it does not recognize, it MUST ignore it.
If the Client receives a DisplayUnshare message with a display-id that it does not recognize, it MUST ignore it.

\begin{center}
    Host \textrightarrow\ Client\\
    \begin{tabular}{|c|c|c|}
        \hline
        \textbf{Bytes} & \textbf{Name} & \textbf{Value} \\
        \hline
        1              & type          & 3              \\
        \hline
        1              & display-id    & 0-255          \\
        \hline
        1              & access        & see below      \\
        \hline
        2              & name-length   & 0-255          \\
        \hline
        name-length    & name          &                \\
        \hline
    \end{tabular}
\end{center}

The access bits are defined below in little endian:

\begin{center}
    \begin{tabular}{|c|c|}
        \hline
        \textbf{Bit} & \textbf{Name} \\
        \hline
        0            & controllable  \\
        \hline
        1-7          & reserved      \\
        \hline
    \end{tabular}
\end{center}

\subsubsection{DisplayShareAck - TCP}

The DisplayShareAck message is sent in reply after receiving a DisplayShare message.
It indicates to the Host, that the Client is ready to receive FrameData.

\begin{center}
    Client \textrightarrow\ Host\\
    \begin{tabular}{|c|c|c|}
        \hline
        \textbf{Bytes} & \textbf{Name} & \textbf{Value} \\
        \hline
        1              & type          & 4              \\
        \hline
        1              & display-id    & 0-255          \\
        \hline
    \end{tabular}
\end{center}

\subsubsection{DisplayUnshare - TCP}

The DisplayUnshare message is sent by the Host to the Client to indicate that the Display is no longer shared.
The Host MUST NOT send FrameUpdates for a Display with this display-id until a new Display with this id has been
reshared via the DisplayShare message.

\begin{center}
    Host \textrightarrow\ Client\\
    \begin{tabular}{|c|c|c|}
        \hline
        \textbf{Bytes} & \textbf{Name} & \textbf{Value} \\
        \hline
        1              & type          & 5              \\
        \hline
        1              & display-id    & 0-255          \\
        \hline
    \end{tabular}
\end{center}

\subsubsection{MouseLocation - TCP/UDP}

The \emph{MouseLocation} message send information about where the mouse is currently on the screen.
The Host sends this information periodically throughout the session.
The Host SHOULD send a \emph{MouseLocation} update when mouse input is received from the Host's system or in
reply when it receives a \emph{MouseInput}.\\

Each shared Display has its own pointer.
Each pointer has a visibility state of visible or hidden.
Visible pointers are drawn by the Client while hidden pointers are not.
Displays begin with a hidden pointer.
When a \emph{MouseLocation} message is received, the pointer for the Display (specified by the display-id field) is made visible.
When a \emph{MouseHidden} message is received, the pointer for the Display is made hidden.

\begin{center}
    Host \textrightarrow\ Client\\
    \begin{tabular}{|c|c|c|c|}
        \hline
        \textbf{Bytes} & \textbf{Name} & \textbf{Value} & \textbf{Description}      \\
        \hline
        1              & type          & 6              &                           \\
        \hline
        1              & display-id    & 0-255          &                           \\
        \hline
        2              & x-location    &                & x coordinate of the mouse \\
        \hline
        2              & y-location    &                & y coordinate of the mouse \\
        \hline
    \end{tabular}
\end{center}

\subsubsection{MouseHidden - TCP/UDP}

\begin{center}
    Host \textrightarrow\ Client\\
    \begin{tabular}{|c|c|c|c|}
        \hline
        \textbf{Bytes} & \textbf{Name} & \textbf{Value} & \textbf{Description} \\
        \hline
        1              & type          & 7              &                      \\
        \hline
        1              & display-id    & 0-255          &                      \\
        \hline
    \end{tabular}
\end{center}

\subsection{Input}

Input messages (including \emph{MouseLocation}) may be sent over TCP or UDP. TCP is preferred in most situations.
However, in situations where speed is prioritized over the guarantees TCP provides (such as gaming), UDP can be
used.

\subsubsection{MouseInput - TCP/UDP}

\begin{center}
    Client \textrightarrow\ Host\\
    \begin{tabular}{|c|c|c|c|}
        \hline
        \textbf{Bytes} & \textbf{Name}     & \textbf{Value} & \textbf{Description}      \\
        \hline
        1              & type              & 8              &                           \\
        \hline
        1              & display-id        & 0-255          &                           \\
        \hline
        2              & x-position        &                & x coordinate of the mouse \\
        \hline
        2              & y-position        &                & y coordinate of the mouse \\
        \hline
        1              & button-mask-delta &                & described below           \\
        \hline
        1              & button-mask-state &                & described below           \\
        \hline
    \end{tabular}
\end{center}

%  https://github.com/rfbproto/rfbproto/blob/master/rfbproto.rst#pointerevent %
Indicates either pointer movement or a pointer button press or release. The pointer is now at (x-position,
y-position), and the current state of buttons 1 to 8 are represented by bits 0 to 7 of button-mask respectively,
0 meaning up, 1 meaning down (pressed).\\

On a conventional mouse, buttons 1, 2 and 3 correspond to the left, middle and right buttons on the mouse. On a
wheel mouse, each step of the wheel is represented by a press and release of a certain button. Button 4 means up,
button 5 means down, button 6 means left and button 7 means right.\\

button-mask-delta indicates which mouse buttons have state updates (1 indicates a state update). button-mask-state have the actual up or down state. Only state updates for buttons indicated by button-mask-delta should be considered.

\subsubsection{KeyInput - TCP/UDP}

The \emph{KeyInput} event sends key presses or releases.

\begin{center}
    Client \textrightarrow\ Host\\
    \begin{tabular}{|c|c|c|c|}
        \hline
        \textbf{Bytes} & \textbf{Name} & \textbf{Value} & \textbf{Description}                                 \\
        \hline
        1              & type          & 9              &                                                      \\
        \hline
        1              & down-flag     & 0 or 1         & indicates whether the key is now pressed or released \\
        \hline
        4              & key           &                & "keysym"                                             \\
        \hline
    \end{tabular}
\end{center}

Details can be found at the \href{https://github.com/rfbproto/rfbproto/blob/master/rfbproto.rst#keyevent}{RFB Spec}

\subsection{Clipboard}

\subsubsection{ClipboardRequest - TCP}

Used to check if a clipboard type exists on the Host.

\begin{center}
    Client \textrightarrow\ Host\\
    \begin{tabular}{|c|c|c|}
        \hline
        \textbf{Bytes}   & \textbf{Name}    & \textbf{Value} \\
        \hline
        1                & type             & 10              \\
        \hline
        1                & clipboard-type   &                \\
        \hline
        \multicolumn{3}{|c|}{\textbf{Below only if clipboard-type's first bit is 1} } \\
        \hline
        1                & type-name-length &                \\
        \hline
        type-name-length & type-name        &                \\
        \hline
    \end{tabular}
\end{center}

clipboard-type first bit (MSB) indicates whether this request is for a default type (\texttt{0}) or a custom type
(\texttt{1}). clipboard-type's second bit (second MSB) indicates whether this request is a exists request (\texttt{0}) or a
content request(\texttt{1}). An exists request is for checking whether the type exists but does not return content. A
content request returns content if it exists. The remaining bits indicate the default type if the request is for a
default type. Otherwise they MUST be 0.\\

clipboard-type's remaining bits referring to the following default types:

\begin{center}
    \begin{tabular}{|c|c|}
        \hline
        \textbf{Value} & \textbf{Description} \\
        \hline
        0              & text                 \\
        \hline
        1              & text                 \\
        \hline
        2              & rtf                  \\
        \hline
        3              & html                 \\
        \hline
        4              & file-pointer         \\
        \hline
    \end{tabular}
\end{center}

\subsubsection{ClipboardNotification- TCP}

Notifies a Peer of a clipboard update. The receiving Peer should update their clipboard.

\begin{center}
    Host $\leftrightarrow$ Client\\
    \begin{tabular}{|c|c|c|}
        \hline
        \textbf{Bytes}        & \textbf{Name}    & \textbf{Value}   \\
        \hline
        1                     & type             & 11               \\
        \hline
        1                     & clipboard-type   &                  \\
        \hline
        \multicolumn{3}{|c|}{\textbf{Below only if clipboard-type's first bit is 1} } \\
        \hline
        1                     & type-name-length &                  \\
        \hline
        type-name-length      & type-name        &                  \\
        \hline
        \multicolumn{3}{|c|}{\textbf{Below always} } \\
        \hline
        1                     & type-exists      & 0 or 1           \\
        \hline
        \multicolumn{3}{|c|}{\textbf{Below only if clipboard-type's second bit is 1 and type-exists is 1} }
        \\
        \hline
        3 & content-length             & \\
        \hline
        \emph{content-length} & data             & zlib'ed raw data \\
        \hline
    \end{tabular}
\end{center}

type-exists indicates whether the clipboard type specified exists on sending Peer.
If the notification is a response to a request type-exists can be 0 to indicate that type was not found, or 1 to indicate that type was found.
If the notification is not a response type-exists SHOULD be 1.
type-exists can be 1 and content-length can be 0 if for example the clipboard text type exists, but is currently empty.

clipboard-type, type-name-length, and type-name MUST match a request if the notification is in response. If
clipboard-readable is 0 and a Host receives a ClipboardRequest, it MUST be ignored. If no Display is Controllable or
clipboard-readable is 0 and a Host receives a ClipboardNotification, it MUST be ignored.\\

content-length -  the length of the content  (maximum $2^{24}$ bytes or ~16MB )\\

\subsection{FrameData - UDP}

The \emph{FrameData} message contains a single RTP-VP9 or RTCP packet

\begin{center}
    Host \textrightarrow Client\\
    \begin{tabular}{|c|c|c|}
        \hline
        \textbf{Bytes} & \textbf{Name} & \textbf{Value} \\
        \hline
        1              & type          & 12             \\
        \hline
        1              & display-id    & 0-255          \\
        \hline
        2              & size          &                \\
        \hline
        \emph{size}    & data          &                \\
        \hline
    \end{tabular}
\end{center}