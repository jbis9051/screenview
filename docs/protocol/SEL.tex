\section{Server Encryption Layer (SEL)}

SEL provides security for communication between Peers and the Server. TCP and UDP have
different security methods. UDP encryption depends on secrets established in the SVSC protocol and therefore can only
be begin after TCP encryption is already established.

\subsection{TCP}

The TCP SEL is heavily based on a simplification of TLS 1.3 as defined in
\href{https://datatracker.ietf.org/doc/html/rfc8446}{RFC8446}.

\subsubsection{PeerHello}

The Peer begins by sending a PeerHello message with their ephemeral public key.

\begin{center}
    Peer \textrightarrow\ Server\\
    \begin{tabular}{|c|c|c|}
        \hline
        \textbf{Bytes} & \textbf{Name} & \textbf{Value} \\
        \hline
        1              & type          & 1              \\
        \hline
        16             & public-key    &                \\
        \hline
    \end{tabular}
\end{center}

\begin{align*}
    &(E_{peer}^{pub},\,E_{peer}^{priv}) := \text{DH-Generate()}\\
    &\text{public-key} := E_{peer}^{pub}
\end{align*}

\subsubsection{ServerHello}

The Server replies with their certificate list. Like TLS, the is a certificate chain. This ensures that a MITM attack
between the Peer and the Server cannot occur. Additionally, the Server sends their ephemeral public key and a
signature. These ephemeral keys ensure perfect forward secrecy.

\begin{center}
    Server \textrightarrow\ Peer\\
    \begin{tabular}{|c|c|c|}
        \hline
        \textbf{Bytes}             & \textbf{Name}       & \textbf{Value} \\
        \hline
        1                          & type                & 2              \\
        \hline
        4                          & length              &                \\
        \hline
        3                          & certificates-length &                \\
        \hline
        \emph{certificates-length} & certificate\_list   &                \\
        \hline
        16                         & public-key          &                \\
        \hline
        variable                   & certificate-verify  &                \\
        \hline
    \end{tabular}
\end{center}

length is the size of the entire message not including the type. While not technically neccessary, it makes this
message easier to parse.\\

certificate\_list is defined in \href{https://datatracker.ietf.org/doc/html/rfc8446#section-4.4.2}{RFC8446
Section-4.4.2}.

\begin{align*}
    & (E_{serv}^{pub},\, E_{serv}^{priv}) := \text{DH-Generate()}\\
    & \text{public-key} := E_{serv}^{pub}
\end{align*}

certificate-verify is defined in \href{https://datatracker.ietf.org/doc/html/rfc8446#section-4.4
.3}{RFC8446 Section-4.4.3} with the following modification. The content that is signed is:\\

\begin{align*}
    \text{content} := \text{``SreenViewServerVerify''}\ ||\ 0\ ||\ E_{serv}^{pub}
\end{align*}

The Client MUST validate all signatures in accordance with the TLS spec.

\subsubsection{Transport Data Key Derivation}

The Server and Client derive their keys and initialize their nonces.

\begin{align*}
    & C_{peer} = \text{DH}(E_{serv}^{pub},\ E_{peer}^{priv})\\
    & C_{serv} = \text{DH}(E_{peer}^{pub},\ E_{serv}^{priv})\\
    & (\mathit{ST}_{peer}^{send} = \mathit{ST}_{serv}^{recv},\ \mathit{ST}_{peer}^{recv} = \mathit{ST}_{serv}^{send}) := \text{KDF}_2(C_{peer} = C_{serv},
    \ \epsilon) \\
    & \mathit{NT}_{peer}^{send} = \mathit{NT}_{serv}^{recv} = \mathit{NT}_{peer}^{recv} = \mathit{NT}_{serv}^{send} := 0
\end{align*}

\subsubsection{Subsequent Messages: Transport Data Messages}

All subsequent messages are encrypted and authenticated. On receiving a message, if authentication fails the message
MUST be dropped.

\begin{center}
    Peer $\leftrightarrow$ Server\\
    \begin{tabular}{|c|c|c|}
        \hline
        \textbf{Bytes}     & \textbf{Name} & \textbf{Value} \\
        \hline
        1                  & type          & 3              \\
        \hline
        2                  & data-length   &                \\
        \hline
        \emph{data-length} & data          &                \\
        \hline
    \end{tabular}
\end{center}

\begin{align*}
    & \text{data} := \text{AEAD}(\mathit{ST}_{m}^{send}, \mathit{NT}_{m}^{send}, P, \epsilon) \\
    & \mathit{NT}_{m}^{send} := \mathit{NT}_{m}^{send} + 1
\end{align*}

Where $P$ is the payload to be transported\\

$\mathit{NT}_{m}$ is an 64 bit counter that MUST NOT wrap. After a transport message is sent, if $\mathit{NT}_{m}$ equals
($2^{64}-1$) the TCP connection MUST be dropped. Subsequent TCP messages MUST NOT be sent. \\

\subsection{UDP}

UDP encryption and authentication rely on the \emph{session-id}, \emph{peer-id} and \emph{peer-key} values
established in a session
(described in \hyperlink{subsection.4.4}{4.4}). The Server (nor the Peer) MUST NOT
process or reply to any messages that don't pass authentication. This prevents an amplification attack.\\

\subsubsection{Transport Data Key Derivation}

The Server and Peer derive keys.

\begin{align*}
    &  G:= \text{session-id}                                                               \\
    &  H := \text{peer-id}                                                                \\
    &  J := \text{peer-key}                                                              \\
    &  (\mathit{SU}_{peer}^{send} = \mathit{SU}_{serv}^{recv}, \mathit{SU}_{peer}^{recv} = \mathit{SU}_{serv}^{send}) := \text{KDF}_2(\text{HASH}(G\,
    ||\, H\,||\, J), \epsilon)                                        \\
    &   \mathit{NU}_{peer}^{send} = \mathit{NU}_{serv}^{recv} = \mathit{NU}_{peer}^{recv} = \mathit{NU}_{serv}^{send} := 0
\end{align*}

\subsubsection{Transport Data Messages}

\begin{center}
    Peer \textrightarrow Server\\
    \begin{tabular}{|c|c|c|}
        \hline
        \textbf{Bytes}                & \textbf{Name}  & \textbf{Value} \\
        \hline
        1                             & type           & 4              \\
        \hline
        16                            & \emph{peer-id} &                \\
        \hline
        8                             & counter        &                \\
        \hline
        $\text{UDP length} - 8$ bytes & data           &                \\
        \hline
    \end{tabular}
\end{center}

\begin{center}
    Server \textrightarrow Peer\\
    \begin{tabular}{|c|c|c|}
        \hline
        \textbf{Bytes}                & \textbf{Name} & \textbf{Value} \\
        \hline
        1                             & type          & 5              \\
        \hline
        8                             & counter       &                \\
        \hline
        $\text{UDP length} - 8$ bytes & data          &                \\
        \hline
    \end{tabular}
\end{center}


\begin{align*}
    & \text{data} := \text{AEAD}(\mathit{SU}_{m}^{send}, \mathit{NU}_{m}^{send}, P, \epsilon)\\
    & \text{counter} := \mathit{NU}_{m}^{send}\\
    & \mathit{NU}_{m}^{send} := \mathit{NU}_{m}^{send} + 1
\end{align*}


Where $P$ is the payload to be transported\\

$\mathit{NU}_{m}$ is an 64 bit counter that MUST NOT wrap. After a transport message is sent, if $\mathit{NU}_{m}$ equals
($2^{64}-1$) the TCP connection MUST be dropped. Subsequent UDP messages MUST NOT be sent. \\

