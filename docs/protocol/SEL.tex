\section{Server Encryption Layer (SEL)}

SEL provides security for communication between Peers and the Server. TCP and UDP have
different security methods. UDP encryption depends on secrets established in the SVSC protocol and therefore can only
be begin after TCP encryption is already established.

\subsection{TCP}

The TCP SEL is TLS 1.3 as defined in
\href{https://datatracker.ietf.org/doc/html/rfc8446}{RFC8446}. TLS v1.3 MUST be used. Previous versions of TLS MUST NOT be used.\\

The Client initiates a TLS connection with the Server in accordance with \href{https://datatracker.ietf.org/doc/html/rfc8446}{RFC8446}. All subsiquent SEL TCP communications occurs over TLS:

\begin{center}
    Peer $\leftrightarrow$ Server\\
    \begin{tabular}{|c|c|c|}
        \hline
        \textbf{Bytes}       & \textbf{Name} & \textbf{Value} \\
        \hline
        2                    & data-length   &                \\
        \hline
        1                    & type          & 1              \\
        \hline
        \emph{data-length-1} & data          &                \\
        \hline
    \end{tabular}
\end{center}

\subsection{UDP}

UDP encryption and authentication rely on the \emph{session-id}, \emph{peer-id} and \emph{peer-key} values
established in a session
(described in \hyperlink{subsection.4.4}{4.4}). The Server (nor the Peer) MUST NOT
process or reply to any messages that don't pass authentication. This prevents an amplification attack.\\

\subsubsection{Transport Data Key Derivation}

The Server and Peer derive keys.

\begin{align*}
    &  G\coloneqq \text{session-id}                                                               \\
    &  H \coloneqq \text{peer-id}                                                                \\
    &  J \coloneqq \text{peer-key}                                                              \\
    &  (\mathit{SU}_{peer}^{send} = \mathit{SU}_{serv}^{recv}, \mathit{SU}_{peer}^{recv} = \mathit{SU}_{serv}^{send}) \coloneqq \text{KDF}_2(\text{HASH}(G\,
    ||\, H\,||\, J), \epsilon)                                        \\
    &   \mathit{NU}_{peer}^{send} = \mathit{NU}_{serv}^{recv} = \mathit{NU}_{peer}^{recv} = \mathit{NU}_{serv}^{send} \coloneqq 0
\end{align*}

\subsubsection{Transport Data Messages}

\begin{center}
    Peer \textrightarrow Server\\
    \begin{tabular}{|c|c|c|}
        \hline
        \textbf{Bytes}  & \textbf{Name}  & \textbf{Value} \\
        \hline
        2               & data-length    &                \\
        \hline
        1               & type           & 2              \\
        \hline
        16              & \emph{peer-id} &                \\
        \hline
        8               & counter        &                \\
        \hline
        remaining bytes & data           &                \\
        \hline
    \end{tabular}
\end{center}

\begin{center}
    Server \textrightarrow Peer\\
    \begin{tabular}{|c|c|c|}
        \hline
        \textbf{Bytes}  & \textbf{Name} & \textbf{Value} \\
        \hline
        2               & data-length   &                \\
        \hline
        1               & type          & 3              \\
        \hline
        8               & counter       &                \\
        \hline
        remaining bytes & data          &                \\
        \hline
    \end{tabular}
\end{center}


\begin{align*}
    & \text{data} \coloneqq \text{AEAD}(\mathit{SU}_{m}^{send}, \mathit{NU}_{m}^{send}, P, \epsilon)\\
    & \text{counter} \coloneqq \mathit{NU}_{m}^{send}\\
    & \mathit{NU}_{m}^{send} \coloneqq \mathit{NU}_{m}^{send} + 1
\end{align*}


Where $P$ is the payload to be transported\\

$\mathit{NU}_{m}$ is an 64 bit counter that MUST NOT wrap. After a transport message is sent, if $\mathit{NU}_{m}$ equals
($2^{64}-1$) the TCP connection MUST be dropped. Subsequent UDP messages MUST NOT be sent. \\

