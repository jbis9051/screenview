\section{Definitions}

The key words "MUST", "MUST NOT", "REQUIRED", "SHALL", "SHALL NOT",
"SHOULD", "SHOULD NOT", "RECOMMENDED", "MAY", and "OPTIONAL" in this
document are to be interpreted as described in \href{https://datatracker.ietf.org/doc/html/rfc2119}{RFC 2119}.

\subsection{Cryptographical Definitions}

Based on \href{https://www.wireguard.com/papers/wireguard.pdf}{Wireguard}:\\


$\epsilon$ represents an empty zero-length bitstring\\
\textbf{\textsc{DH(private key, public key)}} Curve25519 point multiplication of private key and public key, re-
turning 32 bytes of output.\\
\textbf{\textsc{DH-Generate()} }Generates a random Curve25519 private key and derives its corresponding public key,
returning a pair of 32 bytes values, (private, public).\\
\textbf{\textsc{Aead(key, counter, plain text, auth text)}} ChaCha20Poly1305 AEAD, as specified in \href{https://www.rfc-editor.org/rfc/rfc7539}{RFC7539},
with its nonce being composed of 32 bits of zeros followed by the 64-bit little-endian value of counter.\\
\textbf{\textsc{Hash(input)}} BLAKE3(input), returning 32 bytes of output.\\
\textbf{\textsc{Hmac(key, input)}} Hmac-BLAKE3(key, input), the ordinary BLAKE3 hash function used in an
HMAC construction, returning 32 bytes of output.\\
\textbf{\textsc{Kdf$_n$(key, input)}} returns an n-tuple of 32 byte values from \href{https://eprint.iacr.org/2010/264.pdf}{HKDF} with Hash function.