% Preamble
\documentclass[11pt o]{article}

% Packages
\usepackage{amsmath}
\usepackage{textcomp}

\title{Screen View Protocol}
\author{Josh Brown}

\newcommand{\projectName}{Screen View}

% Document
\begin{document}
    \maketitle
    \newpage
    \tableofcontents
    \newpage


    \section{Abstract}
    \projectName{} is an end to end encrypted remote screen viewing and controlling software. This document describes
    the protocols necessary to make it function.

    \newpage


    \section{Definitions}
    The following definitions are used:

    \begin{itemize}
        \item Host - The user that wants to share their screen to the Client
        \item Client - The user that wants to view and control the Host's screen
        \item Server - The intermediary server used for routing and proxying data between
        the Host and the Client
    \end{itemize}


    \section{Frame Protocol}
    The \emph{Frame Protocol} is used to communicate screen data from the Host to the Client.

    \subsection{Initialization}

    \subsection{Frame Data Packet}
    The frame data packet contains an update of a particular grid location.
    \begin{center}
        \begin{tabular}{|c|c|c|}
            \hline
            sequence Number (4 bytes) & size (2 bytes) & Grid Number (2 bytes) \\
            \hline
            \multicolumn{3}{|c|}{data (\emph{size} bytes)} \\
            \hline
        \end{tabular}
    \end{center}

\end{document}
